\documentclass{article}
\usepackage[utf8]{inputenc}
%\usepackage[T1]{fontenc}
\usepackage{amsthm}
\usepackage{amsmath}
\usepackage{amssymb}
\usepackage{mathtools}
\usepackage{graphicx}
\usepackage{hyperref}
\hypersetup{
  colorlinks=true,
  linkcolor=blue,
  filecolor=magenta,
  urlcolor=cyan,
}


\newtheorem*{definition*}{Definition}
\newtheorem*{property*}{Property}
\newcommand\sbullet[1][.5]{\mathbin{\vcenter{\hbox{\scalebox{#1}{$\bullet$}}}}}
\DeclarePairedDelimiter{\norm}{\lVert}{\rVert}
\DeclarePairedDelimiter{\abs}{|}{|}

\usepackage{hyperref}
\hypersetup{
  colorlinks=true,
  linkcolor=blue,
  filecolor=magenta,      
  urlcolor=cyan,
}



\author{phunc20}
\title{Backprop: A Simple Case}
%\date{21 January, 2022}
%\date{January 22, 2022}
\date{January 21, 2021}

\begin{document}

%\maketitle
%\tableofcontents

%\begin{abstract}
%\end{abstract}

\section{Notation de Landau}
Soient $E$ et $F$ deux espace normés sur le corps $\mathbb{K} = \mathbb{R}$ ou $\mathbb{C}$
et $R: E \to F$ une application quelconque. On dit
\begin{itemize}
  \item $R(h)$ est un petit $o$ de norme de $h$ puissance $k$, écrit $R(h) = o(\norm{h}^{k}),$ si
    %$\forall \epsilon > 0 \exists \delta > 0$
    pour tout $\epsilon > 0$ il existe $\delta > 0$ tel que
    $$
      \norm{h}_{E} < \delta \quad\text{entraîne}\quad \norm{R(h)}_{F} \le \epsilon \norm{h}_{E}^{k}
    $$

  \item $R(h)$ est un grand $O$ de norme de $h$ puissance $k$, écrit $R(h) = O(\norm{h}^{k}),$
    s'il existe $C > 0$ et $\delta > 0$ tels que
    $$
      \norm{h}_{E} < \delta \quad\text{entraîne}\quad \norm{R(h)}_{F} \le C \norm{h}_{E}^{k}
    $$
\end{itemize}

En particulier, on a
$$
R(h) = o(\norm{h}^{k}) \implies R(h) = O(\norm{h}^{k})\,
$$
puisqu'il suffit de choisir n'importe quelle paire $(\epsilon, \delta)$ et de la prendre pour $(C, \delta)$.
Ou bien, si l'on se permet de prendre $o(\norm{h}^{k})$ et $O(\norm{h}^{k})$ pour des ensembles des fonctions
satisfaisant les conditions ci-dessus, alors on a
$$
o(\norm{h}^{k}) \subseteq O(\norm{h}^{k})
$$

On parle aussi de petit $o$, grand $O$ pour les fonctions définies sur les entiers positifs.
Cf. \href{https://github.com/phunc20/algorithms/tree/main/CLRS/ch03-growth\_of\_fn/01-asymptotic\_notation}{\texttt{github.com/phunc20/algorithms/.../01-asymptotic\_notation}}

\end{document}
