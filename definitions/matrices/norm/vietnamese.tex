% Mai The Duy's documents
%+ ======================================================

\documentclass[14pt,twoside,a4paper, openany]{book}

\usepackage{indentfirst}
\usepackage{extsizes}
\usepackage{fancyhdr}
\pagestyle{fancyplain}
\pagestyle{fancy}
\usepackage[mathscr]{eucal}
\usepackage{mathrsfs}
\usepackage{makeidx}
\usepackage{mathabx}
%%%%%%%%%%%%%%%%%%%%
\usepackage{boxedminipage}
\usepackage{tkz-tab}
\usepackage{fancybox}
\usepackage{tcolorbox}
\usepackage[colorlinks=true]{hyperref}
\hypersetup{urlcolor=blue}
%%%%%%%%%%%%%%%%%%%%%%%
\usepackage{pgf,tikz}
\usetikzlibrary{arrows}
\pagestyle{empty}
%%%%%%%%%%%%%%%%%%%%

\usepackage[right=2.5cm,left=2.0cm,top=3.5cm,bottom=1.5cm]{geometry} % căn lề
\renewcommand{\baselinestretch}{1.15} % giãn dòng

\usepackage{multirow}
\usepackage{longtable}
%%%%%%%%%
\usepackage[utf8]{vietnam}
\usepackage{color, graphicx}
\usepackage{maybemath}
\usepackage{arcs}
\usepackage{multicol}
\usepackage{wrapfig}
\parskip 2pt
\headsep=12pt
\renewcommand{\headwidth}{13truecm} % chuẩn
\renewcommand{\chaptermark}[1]%
              {\markboth{\it #1}{}}
\renewcommand{\sectionmark}[1]%
           {\markright{\it \thesection\ #1}}
\lhead[\fancyplain{}{\it \thepage}]%
{\fancyplain{}{\rightmark}}
\rhead[\fancyplain{}{\leftmark}]%
   {\fancyplain{}{\it \thepage}}
\cfoot{}
\sloppy

\makeindex


%]-
\theoremstyle{definition}
\newtheorem{definition}{Định nghĩa}[chapter]
\newtheorem{excercise}{Bài}[chapter]
\newtheorem{case}{Trường hợp}[chapter]
\newtheorem{vd}{Ví dụ}[chapter]
\newtheorem{btt}{Bài tập}[chapter]
\newtheorem*{remark}{Nhận xét}
\newtheorem*{note}{Chú ý}
\newtheorem*{pt}{Phân tích}
\newtheorem*{pp}{Phương pháp}
\newtheorem{bt}{Bài toán}

\theoremstyle{plain}
\newtheorem{theorem}{Định lý}[chapter]
\newtheorem{proposition}{Mệnh đề}[chapter]
\newtheorem{lemma}{Bổ đề}[chapter]
\newtheorem{corollary}{Hệ quả}[chapter]
\newtheorem{property}{Tính chất}[chapter]

 \usepackage[tight,vietnam]{minitoc}%
\renewcommand{\mtctitle}{\ } %dinh nghia lai ten
\setcounter{minitocdepth}{1}
%%%%%%%%%%%%%%%%%%%%%%%%%%
\renewcommand{\baselinestretch}{1.2}

\newcommand{\suy}{\Rightarrow}
\newcommand{\kkhi}{\Leftrightarrow}
\newcommand{\vt}[1]{\overrightarrow{#1}}
%%%%%%%%%%%%%%%%%%%%%%%%%%%%%%%%%%%%%%
\newcommand{\hop}[1]{\begin{tcolorbox}[colback=cyan!15,colframe=orange!75!white]{#1}\end{tcolorbox}}
\definecolor{lightgray}{gray}{.95}
%%%%%%%%%%%%%%%%%%%%%%%%%%%%%%%%%%%%%%%
\usepackage{answers}
\Newassociation{Solution1}{Answer}{TotalSolution}
\Newassociation{Solution2}{Answer}{TotalSolution}
\Newassociation{Solution3}{Answer}{TotalSolution}
\Newassociation{Solution4}{Answer}{TotalSolution}

\makeatletter
\providecommand{\dang}[1]
{\noindent\hskip -5pt\textbf{ #1}}
\def\thickhrulefill{\leavevmode \leaders \hrule height 1ex \hfill \kern \z@}
\def\@makechapterhead#1{
\vspace*{10\p@}
{\parindent \z@ \centering \reset@font
%\thickhrulefill\quad
\scshape {\Large {\sf CHƯƠNG \thechapter}}
\quad \hrulefill
\par\nobreak
\vspace*{0\p@}
\interlinepenalty\@M
\thickhrulefill
\par\nobreak
\vspace*{10\p@}
\fontsize{20pt}{15pt} \bfseries \selectfont \scshape\MakeUppercase{#1} \par\nobreak
\par
\vspace*{10\p@}
\hrule
\vskip 30\p@
}}
\def\@makeschapterhead#1{
\vspace*{10\p@}
{\parindent \z@ \raggedright \reset@font
%\thickhrulefill
\par\nobreak
\vspace*{10\p@}
\interlinepenalty\@M
%\hrule
\vspace*{10\p@}
\fontsize{23pt}{15pt} \bfseries {#1}\par\nobreak
\par
\vspace*{5pt}
\hrulefill
\vskip 50\p@
}}
\providecommand\chapter
{\if@openright
\cleardoublepage\else
\clearpage\fi
\thispagestyle{fancy}
\global\@topnum\z@
\@afterindentfalse
\secdef\@chapter\@schapter}

\def\m{\mathscr{M}}
\def\l{\mathscr{L}}
\def\r{\mathscr{R}}
\newcommand\abs[1]{|#1|}
\newcommand\set[1]{\{#1\}}
\newcommand\norm[1]{||#1||}
\DeclareMathOperator{\vrai}{vrai}





\begin{document}
\boldmath
\begin{description}
\item[I] Nhân đa thức với đa thức
\begin{description}
\item[1] Nhân đơn thức với đơn thức
\hop{
Để nhân hai đơn thức, ta nhân các hệ số với nhau và nhân các phần biến với nhau.
}
\textbf{Ví dụ $1$} Tìm tích của
\begin{multicols}{2}
 $x$ và $2xy$\\
$y$ và $-2xy$.
\end{multicols}


\item[2] Nhân đơn thức với đa thức
\hop{
Muốn nhân một đơn thức với một đa thức, ta nhân đơn thức với từng hạng tử của đa thức rồi cộng các tích với nhau.}
\textbf{Ví du 2} Làm tính nhân.
\begin{multicols}{2}
\noindent a) $x(x+y)$\\
b) $y(x-y)$\\
c) $x(x^2+2xy+y^2)$\\
d) $y(x^2+2xy+y^2)$
\end{multicols}


\item[3] Nhân đa thức với đa thức
\hop{Muốn nhân một đa thức với một đa thức, ta nhân mỗi hạng tử của đa thức này với từng hạng tử của đa thức kia rồi cộng các tích với nhau.}
\textbf{Ví dụ $3$} Làm tính nhân
\begin{multicols}{2}
$(x+y)(x-y)$\\
$(x+y)(x^2-xy+y^2)$
\end{multicols}
\end{description}
\item[II] Những hằng đẳng thức đáng nhớ
\begin{description}
\item[$1.$] Bình phương của một tổng.
\begin{equation*}
\begin{split}
(x+y)^{2}&=(x+y)(x+y)\\
                 &=x(x+y)+y(x+y)\\
                   &=(x^2+xy)+(yx+y^2)\\
                &=x^2+xy+yx+y^2\\
                &=x^2+2xy+y^2.
\end{split}
\end{equation*}
Thơ để nhớ :\\
\emph{Bình phương của một tổng\\
Tất cả đều bậc hai\\
Hệ số ở hàng hai\\
Một cộng hai rồi một}\\

Nếu tổng quát :\\
\emph{Mũ n của một tổng\\
Tất cả đều bậc n\\
Hệ số ở hàng n\\
Tam giác Pascal}

\item[$2.$] Bình phương của một hiệu. Áp dụng hằng đẳng thức số $1$ ta có
\begin{equation*}
\begin{split}
(x-y)^2&=[x+(-y)]^2\\
   &=x^2+2x(-y)+(-y)^2\\
     &=x^2-2xy+y^2.
\end{split}
\end{equation*}
Thơ để nhớ :\\
\emph{Bình phương của một hiệu\\
Tất cả đều bậc hai\\
Hệ số ở hàng hai\\
Một trừ hai rồi một}
\item[$3.$] Hiệu hai bình phương
\begin{equation*}
\begin{split}
x^2-y^2&=x^2-xy+xy-y^2\\
                 &=x(x-y)+y(x-y)\\
                   &=(x+y)(x-y).
\end{split}
\end{equation*}
Thơ để nhớ:\\
\emph{Hiệu của hai bình phương\\
Tổng bình thường nhân hiệu\\}

\item[$4.$] Lập phương của một tổng
\begin{equation*}
\begin{split}
(x+y)^3&=(x+y)(x+y)^2\\
                 &=(x+y)(x^2+2xy+y^2)\\
                   &=x^3+3x^2y+3xy^2+y^3\\
                  &=x^3+y^3+3xy(x+y)
\end{split}
\end{equation*}
Thơ để nhớ :\\
\emph{Lập phương của một tổng\\
Tất cả đều bậc ba\\
Hệ số ở hàng ba\\
Một ba rồi ba một}
\item[$5.$] Lập phương của một hiệu. Áp dụng hằng đẳng thức số $4$ bằng cách thay $y$ bằng $-y$ ta được
$$ (x-y)^3=x^3-3x^2y+3xy^2-y^3 $$
Thơ để nhớ :\\
\emph{Lập phương của một hiệu\\
Cũng giống tổng lập phương\\
Nhưng tổng với số âm\\
Nên có dấu xen kẽ}
\item[$6.$] Tổng hai lập phương.

Từ hằng đẳng thức số $4$ suy ra
\begin{equation*}
\begin{split}
x^3+y^3&=(x+y)^3-3xy(x+y)\\
  &=(x+y)[(x+y)^2-3xy]\\
&=(x+y)(x^2+2xy+y^2-3xy)\\
&=(x+y)(x^2-xy+y^2)
\end{split}
\end{equation*}
Thơ để nhớ :\\
\emph{Tổng của hai lập phương\\
Là tích tổng của nó\\
Nhân với bình phương thiếu\\}
\item[$7.$] Hiệu hai lập phương
\begin{equation*}
\begin{split}
x^3-y^3&=x^3+(-y)^3\\
             &=[x+(-y)][x^2-x(-y)+(-y)^2]\\
             &=(x-y)(x^2+xy+y^2)
\end{split}
\end{equation*}
Thơ để nhớ :\\
\emph{Hiệu của hai lập phương\\
Là tích hiệu của nó\\
Nhân với bình phương thiếu\\}
\end{description}

\end{description}


\end{document}


